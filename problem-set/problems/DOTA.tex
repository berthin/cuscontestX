\begin{problem}{Warrencio y el dota}{Standard input}{Standard output}{\textcolor{red}{1} second}{}

% Original idea         
% Problem statement     
% Testset               

Cierto día Warrencio y sus $N$ amigos fueron al BZ a jugar dota. Cada uno tiene preferencia de usar algunas máquinas y no puede usar otras aunque estén disponibles --- al final un jugador sólo puede usar a lo más una máquina pero puede preferir varias. Al llegar a recinto de la sabiduría y las buenas palabras notaron que sólo disponen de m máquinas. Ayuda a Warrencio a calcular cuantos de sus amigos jugarán con él. Warren siempre va a usar una máquina ya que el vicio le hizo tener un trato especial con el dueño del internet.

\InputFile
La entrada consta de $T$ casos de entrada, cada caso sigue el siguiente formato de entrada: La primera línea contiene $3$ enteros $N$, $M $ y $P$. $N$ es el número de amigos de warren, $M$ es el número de máquinas y $P$ es el número de preferencias de todos los jugadores. Las siguiente $P$ líneas contiene $2$ enteros $u,v$. $u$ es el índice de los jugadores, y $v$ es el índice de las máquinas, $0 \leq u \leq n$  y $0 \leq v<m$ (warren tiene el índice $0$) ademas $0 \leq N \leq 99$, $0 \leq M \leq 100$ y $0 \leq P \leq N\times M$.

\OutputFile
Para cada caso se debe imprirmir el número máximo de amigos de warren que tienen una maquina para jugar.  

\Example

\begin{example}
\exmp{%%INPUT
1
3 4 5
0 2
1 0
2 3
3 1
3 2

%%END-INPUT
}{ %%OUTPUT
3
} %%END-OUTPUT
\end{example}

\end{problem}
