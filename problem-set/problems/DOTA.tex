\begin{problem}{Warrencio y el dota}{Standard input}{Standard output}{\textcolor{red}{1} second}{}

% Original idea         
% Problem statement     
% Testset               

Cierto día warrencio y sus $n$ amigos fueron al BZ a jugar dota, cada uno tiene preferencia de usar algunas máquinas y no puede usar otras aunque estén disponibles,(al final un jugador sólo puede usar a lo más una máquina pero puede preferir varias) al llegar a recinto de la sabiduría y las buenas palabras notaron que sólo disponen de m máquinas, ayuda a warrencio a calcular cuantos de sus amigos jugarán con él. (Warren siempre va a usar una máquina ya que el vicio le hizo tener un trato con el dueño del internet).

\InputFile
la entrada consta de $T$ casos de entrada, cada caso sigue el siguiente formato de entrada: 
la primera línea contiene $3$ enteros $n$, $m $ y $ p$. $n$ es el número de amigos de warren, $m$ es el número de máquinas y $p$ es el número de preferencias de todos los jugadores. Las siguiente $p$ líneas contiene $2$ enteros $u,v$. $u$ es el índice de los jugadores, y $v$ es el índice de las máquinas, $0 \leq u \leq n$  y $0 \leq v<m$ (warren tiene el índice $0$)ademas $0 \leq n \leq 99,0 \leq m \leq 100$ y $0 \leq p \leq n*m$.


\OutputFile
para cada caso de se debe imprirmir el número máximo de amigos de warren que tienen una maquina para jugar.  


\Example


\begin{example}
\exmp{%%INPUT
1
3 4 5
0 2
1 0
2 3
3 1
3 2

%%END-INPUT
}{ %%OUTPUT
3
} %%END-OUTPUT
\end{example}

\end{problem}
