\begin{problem}{Warrencio y el Dota}{Standard input}{Standard output}{1 second}{}

% Original idea         
% Problem statement     
% Testset               

Cierto día Warrencio y sus $N$ amigos fueron al BZ a jugar Dota. Cada uno tiene preferencia de usar algunas máquinas y no puede usar otras aunque estén disponibles --- al final un jugador sólo puede usar a lo más una máquina pero puede preferir varias. Al llegar a recinto de la sabiduría y las buenas palabras notaron que sólo disponen de M máquinas. Por lo tanto, ayuda a Warrencio a calcular cuantos de sus amigos jugarán con él. Warren siempre va a usar una máquina ya que el vicio le hizo tener un trato especial con el dueño del internet.

\InputFile
La entrada del problema contiene varios casos de prueba. La primera línea es un entero $T$ indicando el número de casos de prueba. Para cada caso, se sigue el siguiente formato de entrada:

\begin{itemize}
\item La primera línea contiene 3 enteros $N$, $M $ y $P$. $N$ es el número de amigos de Warren, $M$ es el número de máquinas y $P$ es el número de preferencias de todos los jugadores ($0 \leq N \leq 99$, $0 \leq M \leq 100$ y $0 \leq P \leq N\times M$).

\item Las siguientes $P$ líneas contienen 2 enteros $u$ y $v$ siendo $u$ el índice de los jugadores y $v$ es el índice de las máquinas ($0 \leq u \leq N$  y $0 \leq v<M$). Warrencio tiene el índice $0$
\end{itemize}

\OutputFile
Para cada caso se debe imprirmir el número máximo de amigos de Warrencio que tienen una máquina para jugar.  

\Example

\begin{example}
\exmp{%%INPUT
1
3 4 5
0 2
1 0
2 3
3 1
3 2
%%END-INPUT
}{ %%OUTPUT
3
} %%END-OUTPUT
\end{example}

\end{problem}
