\begin{problem}{Nombre del Problema}{Standard input}{Standard output}{\textcolor{red}{TIEMPO} second}{}

% Original idea         
% Problem statement     
% Testset               

Descripci\'on del problema...

\InputFile
El problema contiene varios casos de prueba. La primera l\'inea es un entero $T$ $(1\leq T \leq 10^2)$ que denota el número de casos de prueba. Cada caso está compuesto por dos l\'ineas, en la primera de ellas se encuentran los n\'umeros enteros $N$ y $K$, siendo $N$ el número de piedras medidas ($1 \leq N \leq 10^5$, $1 \leq K \leq 10^9$). Luego, en la segunda l\'inea del caso de prueba se encuentran $N$ enteros distintos $a_1, a_2, \dots, a_N$ que representan las alturas de las piedras ($1 \leq a_i \leq 10^9$).

\OutputFile
Para cada caso de prueba, el programa deber\'a imprimir la longitud del mayor conjunto libre de $K$-m\'ultiplos que se pueda obtener de la lista de alturas.

\Example

\begin{example}
\exmp{%%INPUT
2
2 3
1 3
6 2
2 3 6 5 4 10%%END-INPUT
}{ %%OUTPUT
1
3
} %%END-OUTPUT
\end{example}

\end{problem}
