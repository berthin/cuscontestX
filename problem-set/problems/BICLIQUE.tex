\begin{problem}{Warrencio y los Bicliques}{Standard input}{Standard output}{\textcolor{red}{1} second}{}

% Original idea         
% Problem statement     
% Testset               

El profesor de algoritmos está a punto de desaprobar a Warrencio pero él en la desesperación le propone hacer lo que sea para que pueda aprobar es así que el profesor le pide una tarea titánica el cual consiste en: dado un grafo no dirigido hallar el biclique con más aristas dentro del grafo; al ver la cara entristecida de Warrencio el profesor se apiada  y le da una ayuda, el profesor le da el primer conjunto del biclique, ahora la tarea se resume en: dado un conjunto $S_1$ encontrar el conjunto $S_2$ del biclique para un grafo $G$, se asegura que siempre habrá un $S_2$ de tamaño mayor o igual a $1$. 

Nota:

Un biclique es un grafo bipartido completo, es decir que dado un conjunto de nodos $S_1$ existe un conjuto de nodos $S_2$ tal que existe una arista entre cada par de vertices de $S_1$ y $S_2$.

\InputFile
la entrada consta de $T$ casos de entrada, cada caso sigue el siguiente formato de entrada: 
la primera línea consta de dos  enteros $n$,$m$. $n$ es el número de nodos del grafo no dirigido y $m$ el número de aristas las siguientes $m$ líneas contienen dos enteros $u$ y $v$ que representan una arista entre el nodo $u$ y el nodo $v$ seguido de una un entero $K$ que representa el tamaño de $S_1$ finalmente la última línea de cada caso consta de $K$ enteros que representan los elementos de $S_1$. 

límites:

$1 \leq n,m,k \leq 500; 0 \leq u,v<n $ 

\OutputFile
Para cada caso de prueba, el programa deberá imprimir el conjunto $S_2$.

\Example

\begin{example}
\exmp{%%INPUT
4 4  
1 4
1 3
2 3
2 4
2
3 4%%END-INPUT
}{ %%OUTPUT
1 2
} %%END-OUTPUT
\end{example}

\end{problem}
