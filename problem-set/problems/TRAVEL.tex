\begin{problem}{Nombre del Problema}{Standard input}{Standard output}{\textcolor{red}{1} second}{}

% Original idea         
% Problem statement     
% Testset        
       
Warren y sus amigos piensan ir de paseo hacia Arequipa, para celebrar sus buenas notas en todos los cursos.
Ellos contrataron un omnibus interprovincial para realizar el viaje, hicieron el contrato para viajar a las $5:00$ am y todos debian estar
puntuales. Llegado el día del viaje Warren se quedó dormido así que se apresuro en alistarse, y como es una persona ahorradora
decidió irse en combi pues confiaba que llegaría a tiempo. Cuando ya llegó terminal terrestre, el omnibus al que debia embarcar ya había salido y se encontraba a una distancia $d$ del terminal (todo el trayecto es una linea recta).
Sin embargo aún tiene la opción de realizar el viaje con sus amigos, él puede tomar un taxi que va a una velocidad constante de $v_t$ y alcanzar al omnibus que tiene una velocidad constante de
$v_o$. Warren quiere saber si es posible alcanzar al omnibus y en cuanto tiempo podría alcanzar al bus. Ayuda a Warren con su dilema.


\InputFile
El problema contiene varios casos de prueba. La primera l\'inea es un entero $T$ $(1\leq T \leq 10^4)$ que denota el número de casos de prueba. Cada caso está compuesto UNA l\'inea con los números $v_t,v_o \geq 0$, y $d \geq 0$; siendo $v_t$ la velocidad del taxi, $v_o$ la velocidad del omnibus y $d$ la distancia el omnibus al terminal terrestre.

\OutputFile
Para cada caso de prueba, el programa deber\'a imprimir el tiempo en el cual el taxi puede alcanzar al omnibus con 4 dígitos de precisión, si fuese imposible entonces deber\'a imprimir $-1$.

\Example

\begin{example}
\exmp{%%INPUT
2
2 2 2
5 3 11%%END-INPUT
}{ %%OUTPUT
-1
5.5
} %%END-OUTPUT
\end{example}

\end{problem}
