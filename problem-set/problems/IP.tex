\begin{problem}{Contador de direcciones IP}{Standard input}{Standard output}{1 second}{}

% Original idea         
% Problem statement     
% Testset               

Xavier, un programador muy h\'abil fue contratado por una gran compania de software llamada TORO Out of Control Technologies (TORO-OTC). Reci\'entemente, hab\'ia surgido un problema masivo con las direcciones IP de todas las computadoras de la compan\'ia (aprox. 1000 equipos) y necesitaban con urgencia poder saber la cantidad de direcciones IP que segu\'ian un determinado patr\'on.

Muy brevemente, una direccion IP (la empresa s\'olo trabaja con la versi\'on 4 por el momento)  es una etiqueta asignada a un dispositivo dentro de una red y est\'a representado por un n\'umero de 32 bits. Dicho n\'umero, para ser f\'acilmente leible por nosotros, los humanos, es presentado como 4 n\'umeros decimales (que var\'ian de 0 a 255) separados por un punto. Por ejemplo, algunas direcciones IP son:

\begin{verbatim}
  192.168.1.1   192.168.0.1   172.0.0.1
  192.168.1.13  192.168.0.15  172.0.0.15
  192.168.1.14  192.168.0.16  172.0.0.16
  192.168.1.15  192.168.0.17  172.0.0.17
\end{verbatim}

El problema que TORO-OTC ten\'ia yace b\'asicamente en saber cuantas direcciones fueron asginadas siguiendo un determinado patr\'on. Un patr\'on era una direcci\'on IP que pod\'ia contener asteriscos ('*') los cuales reemplazaban n\'umeros (entre 0 y 9), pero tambi\'en un asterisco puede reemplazar n\'umeros en el rango de 0-255 siempre y cuando no haya ning\'un otro n\'umero que condicione la naturaleza del asterisco --- para mayor claridad, consultar los ejemplos. Considerando las direcciones IP proporcionadas, la Tabla~\ref{tab:ip} muestra algunos ejemplos del tipo de consultas que Xavier deber\'ia de responder.

\begin{table}[h]
\centering
\caption{}
\label{tab:ip}
\begin{tabular}{|l|c|}
\hline
Patrón        & Número de direcciones \\
\hline
\texttt{192.168.1.* }  & 4 \\
\texttt{192.168.*.15}  & 2 \\
\texttt{192.*.*.*   }  & 8 \\
\texttt{192.168.0.14}  & 0 \\
\texttt{192.168.1.1 }  & 1 \\
\texttt{1*2.*.*.1* }   & 6 \\
\texttt{1*2.*.*.*}     & 8 \\
\texttt{*.*.*.*}       & 12 \\ 
\hline
\end{tabular}
\end{table}


\InputFile
El problema contiene varios casos de prueba. Las primera l\'inea es un entero $T$ ($1 \leq T \leq T$) que denota el n\'umero de casos de prueba. Cada caso inicia con una l\'inea ``\texttt{Inicio}'' y termina con una l\'ina ``\texttt{Fin}'' (sin comillas) y está compuesto por una lista de consultas en medio. Las consultas pueden ser de dos tipos. ``\texttt{Agregar IP}'', donde IP tendr\'a el formato de una IP v\'alida antes expuesta, y ``\texttt{Contar PATRON}'' que define el patr\'on sobre el cual se debe de contar las IPs almacenadas hasta el momento. Se garantiza que no se agregar\'an IPs repetidas para un mismo caso de prueba.

\OutputFile
Para cada caso de prueba, el programa deber\'a imprimir ``\texttt{Caso \#i:}'' en una nueva l\'inea (sin comillas) y a continuaci\'ion las respuestas a las consultas. Cada respuesta debe de ir en una l\'inea.

\Example

\begin{example}
\exmp{%%INPUT
2
Inicio
Contar 192.168.1.1
Agregar 192.168.1.1
Agregar 192.168.1.10
Agregar 192.158.1.1
Contar 192.168.1.*
Fin
Inicio
Agregar 172.0.0.1
Agregar 192.0.0.1
Contar 1*2.*.*.1
Contar *.*.*.*
Fin
%%END-INPUT
}{ %%OUTPUT
Caso \#1:
0
2
Caso \#2:
2
2
} %%END-OUTPUT
\end{example}

\end{problem}
