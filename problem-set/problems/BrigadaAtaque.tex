\begin{problem}{Brigada de Ataque}{Standard input}{Standard output}{2 seconds}

% Original idea        
% Problem statement     
% Testset               

El general Fujimori esta planeando su ataque contra su enemigo mortal: el general Kuczynski.

Después de muchos meses de arduo trabajo de inteligencia Fujimori a descubierto que la mejor forma de romper la resistencia enemiga es poner soldados más fuertes al frente de las filas de manera que cada soldado tiene delante suyo a algún compañero que lo supera en fuerza. Formalmente, se tienen $N$ soldados de fuerzas $f_1, f_2, \dots, f_N$ donde $f_i > f_j$ siempre que $1 \leq i < j \leq N$.

Inicialmente los $N$ soldados del ejército forman una fila y debido a que el campo de batalla es accidentado, Fujimori sólo podrá enviar a un pequeño grupo que denominará ``Brigada de Ataque''. Este grupo se formará a partir de la fila original extrayendo algunos soldados para formar una nueva fila, Fujimori recorre la fila de izquierda a derecha y cada vez que desea que un soldado entre en la ``Brigada de Ataque'' lo pone al final de la nueva fila.

Fujimori desea aniquilar a Kuczynski por lo que esta nueva fila debe cumplir con la estructura óptima que fue descubierta por el equipo de inteligencia. Además Fujimori desea que el tamaño de la ``Brigada de Ataque'' sea máximo.

Tú eres el jefe de inteligencia del general Kuczynski y has descubierto el plan de Fujimori, Kuczynski desea saber cuantos soldados enemigos lo atacarán y te encarga esta tarea.

\InputFile

La entrada del problema contiene varios casos de prueba. La primera línea es un entero $T$ ($1\leq T \leq 50$) indicando el número de casos de prueba. Para cada caso, se sigue el siguiente formato de entrada:

\begin{itemize}
\item La primera contiene un número $N$ indicando el tamaño de todo el ejército de Fujimori ($1 \leq N \leq 10^4$).
\item La segunda línea contiene $N$ enteros $f_1, f_2, \dots, f_N$ representando la fuerza de cada soldado de Fujimori ($1 \leq f_i \leq 10^9$).

\end{itemize}

\OutputFile
Para cada caso de prueba se debe imprimir una única línea que contiene el tamaño de la ``Brigada de Ataque'' que debe esperar Kuczynski.

\Example

\begin{example}
\exmp{%%INPUT
2
7
14 15 4 7 3 3 1
5
1 1 1 1 1%%END-INPUT
}{ %%OUTPUT
4
1
} %%END-OUTPUT
\end{example}

~\\ \\
\textbf{\Large{\textsf{Observación}}}

Para el primer caso la ``Brigada de Ataque'' podría estar formada por: 14 7 3 1 ó 15 7 3 1.

Para el segundo caso la ``Brigada de Ataque'' sólo puede estar formada por: 1.

\end{problem}
