\begin{problem}{El viaje de Adam}{Standard input}{Standard output}{1 second}{}

% Original idea         
% Problem statement     
% Testset               

Adam es una persona a quien le gusta conocer nuevos lugares, actualmente él se encuentra en la ciudad $1$ y quiere llegar a la ciudad de $N + 1$ pasando por $N$ sitios durante su recorrido. 
Adam sabe que para ir de la ciudad $i$  a la ciudad $i + 1$ existen $a_i$ formas de llegar. Adam desea saber el número de caminos distintos de llegar a la ciudad $N + 1$ desde la ciudad $1$. 
Sin embargo Adam solo desea saber el primer y último dígito del número de caminos posibles, ya que el número de caminos puede ser un número muy grande.

\InputFile

La entrada del problema contiene varios casos de prueba. La primera línea es un entero $T$ ($1 \leq T \leq 10^3$) indicando el número de casos de prueba. Para cada caso, se sigue el siguiente formato de entrada:

\begin{itemize}
\item La primera línea contiene un entero $N$ indicando el número de ciudades por las cuales Adam debe de pasar ($1 \leq N \leq 10^3$).
\item La segunda línea contiene $N$ enteros distintos $a_1, a_2, \dots, a_N$ representando el número de caminos posibles para ir de la ciudad $i$ a la ciudad $i+1$ ($1 \leq a_i \leq 10^9$).
\end{itemize}

\OutputFile
Para cada caso de prueba, el programa deber\'a imprimir el primer y último dígito de número de caminos posibles para ir de la ciudad $1$ a la ciudad $N$.

\Example

\begin{example}
\exmp{%%INPUT
3
1
10
3
3 5 7
2
123 456%%END-INPUT
}{ %%OUTPUT
1 0
1 5
5 8
} %%END-OUTPUT
\end{example}

\end{problem}
