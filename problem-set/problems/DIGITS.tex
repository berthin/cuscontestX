\begin{problem}{El viaje de Adam}{Standard input}{Standard output}{\textcolor{red}{1} second}{}

% Original idea         
% Problem statement     
% Testset               

Adam es una persona a quien le gusta conocer nuevos lugares, actualmente él se encuentra en la ciudad $1$ y quiere llegar a la ciudad de $n + 1$ pasando por $n$ sitios durante su recorrido. 
Adam sabe que para ir de la ciudad $i$  a la ciudad $i + 1$ existen $a_i$ formas de llegar. Adam desea saber el número de caminos distintos de llegar a la ciudad $n + 1$ desde la ciudad $1$. 
Sin embargo Adam solo desea saber el primer y último dígito del número de caminos posibles, ya que el número de caminos puede ser un número muy grande.

\InputFile
El problema contiene varios casos de prueba. La primera l\'inea es un entero $T$ $(1\leq T \leq 10^3)$ que denota el número de casos de prueba. 
Cada caso está compuesto por dos l\'ineas, en la primera de ellas se encuentran el entero $n$, el cual denora el número de ciudades que adam debe recorrer ($1 \leq n \leq 10^3$). 
Y en la segunda l\'inea del caso de prueba se encuentran $n$ enteros distintos $a_1, a_2, \dots, a_n$ que representan el número de caminos posibles ($1 \leq a_i \leq 10^9$) para ir de la ciudad $i$ a la ciudad $i + 1$.

\OutputFile
Para cada caso de prueba, el programa deber\'a imprimir el primer y último dígito de número de caminos posibles para ir de la ciudad $1$ a la ciudad $n$.

\Example

\begin{example}
\exmp{%%INPUT
3
1
10
3
3 5 7
2
123 456%%END-INPUT
}{ %%OUTPUT
1 0
1 5
5 8
} %%END-OUTPUT
\end{example}

\end{problem}
