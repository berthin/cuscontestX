\providecommand{\abs}[1]{\lvert#1\rvert}
\begin{problem}{Asignatura de Matemáticas}{Standard input}{Standard output}{1 second}

% Original idea        
% Problem statement     
% Testset               

El pequeño Albus estubo haciendo su tarea de matemáticas en la escuela y como es nuevo en esta asignatura practicó lo más básico: las operaciones aritméticas de suma, resta, multiplicación y división.\\
Albus tiene cuatro pedazos largos de papel, para cada problema que resuelve hace los siguiente: En el primer pedazo de papel escribe el primer operando, en el segundo pedazo pone el operador (+, -, *, /), en el tercer pedazo pone el segundo operando y en el cuarto pedazo escribe el resultado de la operación.\\
Lily, la hermana menor de Albus, estubo jugando y quemo el segundo pedazo de papel por accidente. Ella no sabe nada de matemáticas, solo sabe que estará en grandes problemas si su hermano descubre lo que hizo.\\
James osea tú, el hermano mayor de Albus y Lily, debes ayudar a solucionar este problema. Tienes los dos operandos y la respuesta de cada operación, debes determinar el operando perdido.

\InputFile
La entrada estara dada por varios casos de prueba, cada caso de prueba se describirá en una línea formada por tres enteros $x, y, z$ $(0\leq \abs{x}, \abs{y} \leq 100, 0\leq \abs{z} \leq 10000)$. El último caso de prueba estará formado por tres ceros, este caso no debe ser procesado.

\OutputFile
Para cada caso se debe imprimir una línea que contenga ``+'' (sin comillas) para la suma, ``-'' (sin comillas) para la resta, ``*'' (sin comillas) para la multiplicación y ``/'' (sin comillas) para la división.\\
Considere división entera e.g. $ 4 / 3 = 1 $. En caso de existir más de un operador válido imprima el de menor presedencia. Esto es: $ presedencia(-) < presedencia(+) < presedencia(*) < presedencia(/)$\\
Se garantiza que siempre existirá al menos un operador válido para los operandos.

\Example

\begin{example}
\exmp{%%INPUT
12 5 2
2 2 4
1 4 4
4 8 -4
0 0 0%%END-INPUT
}{ %%OUTPUT
/
+
*
-
} %%END-OUTPUT
\end{example}

\end{problem}
