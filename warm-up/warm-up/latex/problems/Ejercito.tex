\begin{problem}{Brigada de Ataque}{Standard input}{Standard output}{2 seconds}

% Original idea        
% Problem statement     
% Testset               

El general Fujimori esta planeando su ataque contra su enemigo mortal: el general Kuczynski.\\
Despues de muchos meses de arduo trabajo de inteligencia Fujimori a descubierto que la mejor forma de romper la resistencia enemiga es poner soldados más fuertes al frente de las filas de manera que cada soldado tiene delante suyo a algun compañero que lo supera en fuerza,  mas formalmente $f_{i} > f_{j}$, $(1 \leq i < j \leq n)$ donde $f_{i}$ indica la fuerza del $i-esimo$ soldado y $n$ es el tamaño total del ejercito de Fujimori.\\
Inicialmente los $n$ soldados del ejército forman una fila y debido a que el campo de batalla es accidentado, Fujimori solo podra enviar a un pequeño grupo que denominará ``Brigada de Ataque'', este grupo se formará a partir de la fila original extrayendo algunos soldados para formar una nueva fila, Fujimori recorre la fila de izquierda a derecha y cada vez que desea que un soldado entre en la ``Brigada de Ataque'' lo pone al final de la nueva fila.\\
Fujimori desea aniquilar a Kuczynski por lo que esta nueva fila debe cumplir con la estructura óptima que fue descubierta por el equipo de inteligencia. Ademas Fujimori desea que el tamaño de la ``Brigada de Ataque'' sea máximo.\\
Tú eres el jefe de inteligencia del general Kuczynski y has descubierto el plan de Fujimori, Kuczynski desea saber cuantos soldados enemigos lo atacarán y te encarga esta tarea.

\InputFile
La entrada empieza con un número $t$ $(1\leq t \leq 50)$, el número total de casos de prueba.\\
Cada caso de prueba estará formado por dos lineas, la primera contiene $n$ $(1\leq n \leq 10^{4})$: el tamaño de todo el ejército de Fujimori.\\
La segunda línea contiene $n$ enteros $f_{i}$ $(1 \leq f_{i} \leq 10^{9})$ , la fuerza de cada soldado de Fujimori, separados por espacios en blanco.

\OutputFile
Para cada caso de prueba se debe imprimir una única línea que contiene el tamaño de la ``Brigada de Ataque'' que debe esperar Kuczynski.

\Example

\begin{example}
\exmp{%%INPUT
2
7
14 15 4 7 3 3 1
5
1 1 1 1 1%%END-INPUT
}{ %%OUTPUT
4
1
} %%END-OUTPUT
\end{example}

Para el primer caso la ``Brigada de Ataque'' podria estar formada por: 14 7 3 1 ó 15 7 3 1.\\
Para el segundo caso la ``Brigada de Ataque'' solo puede estar formada por: 1.
\end{problem}
