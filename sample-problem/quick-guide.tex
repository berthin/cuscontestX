\documentclass[11pt,a4paper,oneside]{article}

\usepackage[english]{babel}
\usepackage{olymp}
\usepackage{amsmath}
%\usepackage{bnf}
\usepackage{graphicx}
%\usepackage{subcaption}
\usepackage{subfig}
\usepackage{wrapfig}
\usepackage{xcolor}

\usepackage{fancyvrb}
\usepackage[utf8]{inputenc}

% Default fixed font does not support bold face
\DeclareFixedFont{\ttb}{T1}{txtt}{bx}{n}{12} % for bold
\DeclareFixedFont{\ttm}{T1}{txtt}{m}{n}{12}  % for normal

% Custom colors
\usepackage{color}
\definecolor{deepblue}{rgb}{0,0,0.5}
\definecolor{deepred}{rgb}{0.6,0,0}
\definecolor{deepgreen}{rgb}{0,0.5,0}

\usepackage{listings}
% Python style for highlighting
\lstdefinestyle{pythonstl}{
language=Python,
basicstyle=\ttm,
otherkeywords={self},             % Add keywords here
keywordstyle=\ttb\color{deepblue},
emph={MyClass,__init__},          % Custom highlighting
emphstyle=\ttb\color{deepred},    % Custom highlighting style
stringstyle=\color{deepgreen},
%frame=tb,                         % Any extra options here
%frame=L,
numberstyle=\color{red},
numbers=left,                    
numbersep=15pt,
xleftmargin=1cm,
showstringspaces=false            % 
}

%\usepackage[font=small,textfont=it]{caption}

\begin{document}

\input SumandoNumeros.tex

\newpage
\textbf{\Large{\textsf{Solution}}}

Solución en Python:
\lstinputlisting[language=Python, style=pythonstl]{SumandoNumeros.py}

\textbf{\Large{\textsf{Judging Stage}}}

El programa enviado al \textbf{juez} será evaluado con casos diversos casos de prueba  y la salida será evaluada con el archivo de salida correcto que se encuentra cargado en el \textbf{juez}.

Por ejemplo, para el problema anterior se han cargado los archivos \texttt{Judge.in} y \texttt{Judge.out} indicando los datos de entrada y salida. El \textbf{juez} compilará el programa enviado y lo ejecutará con \texttt{Judge.in}, la salida \texttt{Answer.out} será comparada con \texttt{Judge.out} y si ambas salidas son iguales, el \textbf{juez} retornará un mensaje de \textbf{Accepted}.


\begin{examplejudge}
\exmpjudge{
10
10
896 545 866 525 713 764 265 159 473 284
9
528 489 971 691 244 222 526 81 609
5
839 480 486 26 507
9
941 97 399 555 330 212 734 834 756
9
206 671 656 596 708 305 282 0 636
6
547 91 120 942 48 610
8
756 584 554 663 474 309 941 495
3
456 390 665
9
727 138 242 873 37 883 984 63 661
5
900 972 283 488 637
}{
Caso \#1: 5490
Caso \#2: 4361
Caso \#3: 2338
Caso \#4: 4858
Caso \#5: 4060
Caso \#6: 2358
Caso \#7: 4776
Caso \#8: 1511
Caso \#9: 4608
Caso \#10: 3280
}{
Caso \#1: 5490
Caso \#2: 4361
Caso \#3: 2338
Caso \#4: 4858
Caso \#5: 4060
Caso \#6: 2358
Caso \#7: 4776
Caso \#8: 1511
Caso \#9: 4608
Caso \#10: 3280
}%
\end{examplejudge}

\end{document}
